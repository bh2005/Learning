
% Setting up the document class and essential packages
\documentclass[a4paper,12pt]{article}

% Including standard LaTeX packages for encoding, fonts, and formatting
\usepackage[utf8]{inputenc}
\usepackage[T1]{fontenc}
\usepackage{lmodern}
\usepackage[ngerman,english]{babel}
\usepackage{csquotes}
\usepackage{tocloft}
\usepackage{geometry}
\geometry{a4paper, margin=2.5cm}
\usepackage{setspace}
\onehalfspacing
\usepackage{graphicx}
\usepackage{caption}
\usepackage{acronym}
\usepackage{bibentry}
\nobibliography*

% Configuring the table of contents
\setlength{\cftsecindent}{0em}
\setlength{\cftsubsecindent}{2em}

% Setting reliable font (Latin Modern for Latin characters)
\usepackage{lmodern}

\begin{document}

% Creating the title page
\begin{titlepage}
    \centering
    \vspace*{1cm}
    {\huge \textbf{[Titel der Arbeit]}}
    \vspace{0.5cm}
    \newline
    {\Large Bachelorarbeit}
    \vspace{1cm}
    \newline
    Hochschule: [Name der Hochschule]\\
    Studiengang: [Name des Studiengangs]\\
    Name: [Dein Name]\\
    Matrikelnummer: [Deine Matrikelnummer]\\
    Betreuer: [Name des Betreuers]\\
    Datum: \today
\end{titlepage}

% Adding abstract
\section*{Abstract}
\selectlanguage{english}
[Hier beschreibst du Thema, Untersuchung, Ergebnisse und Bedeutung in max. 1 Seite.]

% Adding Vorwort
\section*{Vorwort}
\selectlanguage{ngerman}
[Persönlicher Hintergrund der Arbeit.]

% Adding Danksagung
\section*{Danksagung}
[Dank an Familie, Betreuer, Kollegen, etc.]

% Generating table of contents
\tableofcontents
\clearpage

% Generating list of figures
\listoffigures
\clearpage

% Generating list of tables
\listoftables
\clearpage

% Adding Abkürzungsverzeichnis
\section*{Abkürzungsverzeichnis}
\begin{acronym}[API]
    \acro{API}{Application Programming Interface}
    \acro{VIM}{Vi Improved}
\end{acronym}
\clearpage

% Starting main content
\section{Einleitung}
[Vorstellung des Themas, Ziel, Relevanz.]

\section{Theoretischer Teil}
[Schlüsselbegriffe, Theorien, Literaturübersicht.]

\section{Methodik}
[Beschreibung der Forschungsmethoden, z. B. Literatur-Review.]

\section{Ergebnisse}
[Untersuchungsergebnisse, Zuordnung zu Hypothesen.]

\section{Diskussion}
[Analyse, Limitationen, Zukunftsempfehlungen.]

\section{Fazit}
[Zusammenfassung der Ergebnisse, Bezug zur Einleitung.]

% Adding Nachwort
\section*{Nachwort}
[Reflexion über die Arbeit, optional.]

% Adding Literaturverzeichnis
\section*{Literaturverzeichnis}
\begin{thebibliography}{9}
    \bibentry{doe2023}
    \bibitem{doe2023}
        Doe, John (2023). \emph{Introduction to Vim}. Tech Press.
\end{thebibliography}

% Adding Anhang
\appendix
\section{Anhang}
[Ergänzende Materialien, optional.]

% Adding eidesstattliche Erklärung
\section*{Eidesstattliche Erklärung}
Hiermit erkläre ich, dass ich die vorliegende Arbeit selbstständig verfasst und keine anderen als die angegebenen Quellen verwendet habe.

\end{document}
